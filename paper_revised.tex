\documentclass[12pt]{article}
\usepackage[margin=1in]{geometry}
\usepackage{amsmath, amssymb}
\usepackage{graphicx}
\usepackage{booktabs}
\usepackage{caption}
\usepackage{subcaption}
\usepackage{hyperref}
\usepackage{natbib}
\usepackage{setspace}
\usepackage{float}

% Formatting
\doublespacing
\setlength{\parindent}{0.5in}
\setlength{\parskip}{0pt}

% Title and author
\title{Sentiment-Augmented Implied Volatility Forecasting: \\ A PCA-Augmented HAR Approach}
\author{Parker Jahn}
\date{February 2026}

\begin{document}

\maketitle

\begin{abstract}
\noindent This paper investigates whether news sentiment contains predictive information for implied volatility (VIX) beyond traditional autoregressive benchmarks. We augment the standard Heterogeneous Autoregressive model of Implied Volatility (HAR-IV) with PCA-extracted sentiment factors derived from semiconductor news coverage. Using a rolling PCA methodology that avoids look-ahead bias, we find that sentiment significantly improves 1-day ahead VIX forecasts, reducing out-of-sample RMSE by 14.97\% (Diebold-Mariano $p < 0.001$). However, the improvement diminishes at weekly horizons, where HAR-IV persistence dominates, showing slight underperformance that is statistically insignificant. We emphasize methodological rigor: all inference uses horizon-adjusted variance estimators, and we make no causal claims given non-significant Granger causality tests. Our results suggest that sentiment provides modest but statistically significant predictive value for short-horizon volatility trading.

\vspace{0.2in}
\noindent \textbf{Keywords}: Implied Volatility, VIX, Sentiment Analysis, PCA, HAR Model, Volatility Forecasting
\end{abstract}

\newpage

\section{Introduction}

\subsection{Research Question}

Can news sentiment improve out-of-sample forecasts of implied volatility beyond traditional autoregressive models? Specifically, we ask whether sentiment extracted from semiconductor industry news---a sector at the intersection of technology innovation and macroeconomic uncertainty---contains incremental predictive information for the CBOE Volatility Index (VIX).

\subsection{Motivation}

The VIX, often called the ``fear gauge,'' reflects market expectations of near-term volatility and is a critical input for options pricing, risk management, and portfolio allocation. While the Heterogeneous Autoregressive model of Implied Volatility (HAR-IV) of \citet{corsi2009} has become a standard benchmark, capturing volatility persistence at daily, weekly, and monthly frequencies, it relies exclusively on past volatility levels.

Recent advances in natural language processing have enabled systematic extraction of sentiment from financial news. If market participants react to news before volatility materializes, sentiment may contain leading information not captured by historical volatility patterns. We restrict sentiment analysis to January 2023 onwards, coinciding with increased market attention to AI technologies following the release of ChatGPT. This post-AI period allows us to focus on a recent regime while maintaining sufficient sample size for out-of-sample validation.

\subsection{Headline Results}

Our analysis yields three main findings:

\begin{enumerate}
    \item \textbf{1-Day Horizon}: Adding PCA-extracted sentiment factors to HAR-IV reduces RMSE by \textbf{14.97\%} ($p < 0.001$). This is our headline result---sentiment provides statistically and economically significant short-term predictive value.
    
    \item \textbf{5-Day Horizon}: The sentiment-augmented model shows slight underperformance, statistically insignificant ($-1.74\%$, $p = 0.477$). At weekly horizons, HAR-IV persistence dominates.
    
    \item \textbf{22-Day Horizon}: Modest improvement of 1.72\% ($p = 0.051$), marginally significant. Slow-moving sentiment trends may capture structural shifts in market uncertainty.
\end{enumerate}

We emphasize that Granger causality tests are not significant; we make no causal claims about sentiment driving volatility. Our contribution is demonstrating predictive, not causal, relationships.

\section{Data and Methodology}

\subsection{Data Sources}

We construct our dataset from three categories of variables:

\paragraph{Target Variable}
\begin{itemize}
    \item \textbf{VIX}: CBOE Volatility Index, obtained from Yahoo Finance. The VIX represents 30-day expected volatility of S\&P 500 options.
\end{itemize}

\paragraph{Market Factors}
\begin{itemize}
    \item \textbf{SMH, SOXX}: Semiconductor ETFs, used to construct realized volatility and return factors
    \item \textbf{Gold, Copper, Oil}: Commodity returns to capture macro risk factors
\end{itemize}

\paragraph{Sentiment Variables}
\begin{itemize}
    \item \textbf{AlphaVantage Sentiment (AV)}: Proprietary sentiment scores for semiconductor companies (NVDA, AMD, INTC, TSM, MU)
    \item \textbf{FinBERT Sentiment (FB)}: Transformer-based sentiment classification applied to news headlines \citep{araci2019finbert}
\end{itemize}

\subsection{Sample Construction}

Table~\ref{tab:sample} summarizes the data sources and sample construction. While sentiment data collection begins in January 2023, the rolling window training requirement (200 days) and variable lagging push the first actionable forecast date to May 2023, yielding 637 trading-day observations. The 86-observation difference between raw sentiment (723 obs) and final sample (637 obs) arises from: (1) 200-day minimum training window requirement, (2) variable lagging (1-day lags for sentiment), and (3) alignment of weekend observations to trading days only.

\begin{table}[H]
\centering
\caption{Sample Construction}
\label{tab:sample}
\begin{tabular}{llcc}
\toprule
Variable & Source & Period & Observations \\
\midrule
VIX & Yahoo Finance & 2022-01-03 to 2026-01-30 & 1,025 \\
SMH/SOXX & Yahoo Finance & 2022-01-03 to 2026-01-30 & 1,025 \\
AV Sentiment & AlphaVantage API & 2023-01-01 to 2026-01-30 & $\sim$750 \\
FB Sentiment & ProsusAI/finbert & 2023-01-01 to 2026-01-30 & $\sim$750 \\
\textbf{Final Sample} & Merged & 2023-05-08 to 2026-01-30 & \textbf{637} \\
\bottomrule
\end{tabular}
\end{table}

\paragraph{Post-AI Era Restriction} We restrict sentiment analysis to January 2023 onwards, coinciding with increased market attention to AI technologies. This post-AI period allows us to focus on a recent regime while maintaining sufficient sample size for out-of-sample validation.

\paragraph{Weekend Handling} Weekend sentiment observations are excluded. Sentiment is aligned to trading days only to ensure proper synchronization with VIX.

\subsection{Sentiment Residualization}

Raw sentiment may be contemporaneously correlated with returns, creating identification challenges. We orthogonalize sentiment to same-day returns using an expanding-window regression:

\begin{equation}
\text{Shock}_t = \text{Sentiment}_t - \hat{\beta} \cdot \text{Return}_t
\end{equation}

where $\hat{\beta}$ is estimated on data up to time $t-1$ only. Residualization parameters are estimated using expanding windows up to $t-1$, independent of the 200-day rolling CV window. Specifically, we use SMH ETF returns for residualization. Both AlphaVantage and FinBERT sentiment are residualized separately. This ``residualized shock'' isolates sentiment information not already reflected in prices.

\subsection{Rolling PCA}

To extract common factors from our feature set while avoiding look-ahead bias, we implement rolling PCA following the concerns raised by \citet{rossi2011}:

\begin{enumerate}
    \item At each cross-validation fold, fit StandardScaler and PCA on \textbf{training data only}
    \item Transform both training and test data using training-fitted parameters
    \item Report loadings as \textbf{averages across rolling windows} ($\pm$ standard deviation)
\end{enumerate}

PCA is re-estimated at each cross-validation fold (every 25 days) using only the 200-day training window. This ensures that PCA loadings used for forecasting are not contaminated by future information.

We retain 3 principal components, which collectively explain approximately 80\% of the variance in the feature set. This choice balances dimensionality reduction with information preservation.

\paragraph{Component Interpretation} (from rolling-averaged loadings):
\begin{itemize}
    \item \textbf{PC1}: Captures semiconductor return co-movement (SMH/SOXX returns load at $\sim$0.15)
    \item \textbf{PC2}: Captures volatility dynamics (SMH/SOXX RV load at $\sim$0.62)
    \item \textbf{PC3}: Captures sentiment variation (AV/FB sentiment load at $\sim-0.54$)
\end{itemize}

The emergence of sentiment as a distinct factor (PC3) confirms that sentiment contains information orthogonal to returns and volatility. Figure~\ref{fig:pca_loadings} visualizes these loadings.

\subsection{Forecasting Models}

\paragraph{HAR-IV (Baseline)}

\begin{equation}
VIX_{t+h} = \alpha + \beta_1 VIX_{t-1} + \beta_2 VIX^{(w)}_{t-1} + \beta_3 VIX^{(m)}_{t-1} + \varepsilon_{t+h}
\end{equation}

where $VIX^{(w)}$ and $VIX^{(m)}$ are 5-day and 22-day rolling averages of lagged VIX. Note that the 22-day monthly component uses only data available up to $t-1$, ensuring no forward-looking bias even when sentiment data begins in 2023.

\paragraph{HAR-IV + PCA + Sentiment}

\begin{multline}
VIX_{t+h} = \alpha + \beta_1 VIX_{t-1} + \beta_2 VIX^{(w)}_{t-1} + \beta_3 VIX^{(m)}_{t-1} \\
+ \gamma_1 PC1_t + \gamma_2 PC2_t + \gamma_3 PC3_t + \delta_1 AV_{t-1} + \delta_2 FB_{t-1} + \varepsilon_{t+h}
\end{multline}

Both models are estimated via Ridge regression ($\alpha = 1.0$) to stabilize coefficient estimates in the presence of multicollinearity.

\subsection{Cross-Validation Design}

We employ rolling-window cross-validation:
\begin{itemize}
    \item \textbf{Minimum training window}: 200 observations
    \item \textbf{Test window}: 50 observations
    \item \textbf{Step size}: 25 observations
\end{itemize}

This yields approximately 10 folds per horizon, with out-of-sample predictions totaling $\sim$500 observations per horizon.

\subsection{Evaluation Metrics}

\paragraph{Primary Metric} Root Mean Squared Error (RMSE) in VIX points

\paragraph{Statistical Inference} Diebold-Mariano test \citep{diebold1995} with horizon-adjusted Newey-West variance \citep{newey1987}:
\begin{itemize}
    \item Squared error loss function
    \item HAC variance with Bartlett kernel
    \item Lag truncation = $h - 1$ for $h$-step ahead forecasts
\end{itemize}

\section{Results}

\subsection{Stationarity}

All variables pass Augmented Dickey-Fuller tests at the 5\% level (Table~\ref{tab:stationarity}). Sentiment shocks (residualized) are stationary by construction. \textbf{No differencing is required.}

\begin{table}[H]
\centering
\caption{Stationarity Tests (Augmented Dickey-Fuller)}
\label{tab:stationarity}
\begin{tabular}{lccc}
\toprule
Variable & ADF Statistic & $p$-value & Stationary \\
\midrule
VIX & $-5.095$ & $<0.0001$ & Yes \\
SMH Return & $-14.882$ & $<0.0001$ & Yes \\
SMH Volatility & $-3.837$ & 0.0030 & Yes \\
AV Sentiment ($t-1$) & $-3.344$ & 0.0130 & Yes \\
FB Sentiment ($t-1$) & $-2.970$ & 0.0380 & Yes \\
\bottomrule
\end{tabular}
\end{table}

\subsection{PCA Loadings}

The rolling-averaged PCA loadings confirm that sentiment emerges as a distinct factor (Table~\ref{tab:pca_loadings}). The low standard deviations indicate that loadings are generally stable across rolling windows, though with increased volatility in recent folds, supporting the reliability of our factor interpretation.

\begin{table}[H]
\centering
\caption{Rolling-Averaged PCA Loadings (Mean $\pm$ SD)}
\label{tab:pca_loadings}
\begin{tabular}{lccc}
\toprule
Feature & PC1 (Returns) & PC2 (Volatility) & PC3 (Sentiment) \\
\midrule
SMH Return & $-0.15 \pm 0.54$ & $-0.09 \pm 0.26$ & $0.25 \pm 0.03$ \\
SMH Volatility & $0.26 \pm 0.11$ & $0.62 \pm 0.05$ & $0.00 \pm 0.12$ \\
AV Sentiment & $-0.01 \pm 0.17$ & $-0.05 \pm 0.09$ & $\mathbf{-0.54 \pm 0.04}$ \\
FB Sentiment & $-0.04 \pm 0.12$ & $-0.11 \pm 0.04$ & $\mathbf{-0.54 \pm 0.02}$ \\
\bottomrule
\end{tabular}
\end{table}

\subsection{Forecast Performance}

Table~\ref{tab:forecast_performance} presents the main forecasting results. The sentiment-augmented model achieves a 14.97\% reduction in 1-day RMSE, with a Diebold-Mariano $p$-value below 0.001. This is both statistically and economically significant---a reduction from 2.34 to 1.99 VIX points in forecast error.

At the 5-day horizon, the augmented model shows slight underperformance ($-1.74\%$), which is statistically insignificant ($p = 0.477$). The 22-day horizon shows modest improvement (1.72\%) that is marginally significant ($p = 0.051$).

\begin{table}[H]
\centering
\caption{Forecast Performance and Diebold-Mariano Tests}
\label{tab:forecast_performance}
\begin{tabular}{llcccc}
\toprule
Model & Horizon & RMSE & \% Improvement & DM $p$-value & Significant \\
\midrule
HAR-IV & 1-Day & 2.338 & --- & --- & --- \\
HAR-IV+PCA+Sent & 1-Day & 1.988 & $+14.97\%$ & $<0.0001$ & Yes$^{***}$ \\
HAR-IV & 5-Day & 4.350 & --- & --- & --- \\
HAR-IV+PCA+Sent & 5-Day & 4.425 & $-1.74\%$ & 0.4769 & No \\
HAR-IV & 22-Day & 6.506 & --- & --- & --- \\
HAR-IV+PCA+Sent & 22-Day & 6.394 & $+1.72\%$ & 0.0510 & Marginal \\
\bottomrule
\end{tabular}
\begin{tablenotes}
\small
\item $^{***}$ Significant at $\alpha = 0.001$. DM tests use horizon-adjusted Newey-West variance with lag = $h-1$.
\item All tests based on $\sim$500 out-of-sample observations per horizon.
\end{tablenotes}
\end{table}

\subsection{Granger Causality}

For completeness, we report Granger causality test results in Table~\ref{tab:granger}. None of the sentiment variables Granger-cause VIX at conventional significance levels. This confirms that we make no causal claims---our contribution is to forecasting accuracy, not causal identification.

\begin{table}[H]
\centering
\caption{Granger Causality Tests (Sentiment $\rightarrow$ VIX)}
\label{tab:granger}
\begin{tabular}{lccc}
\toprule
Variable & Lag Order & F-Statistic & $p$-value \\
\midrule
AV Sentiment & 5 & 1.23 & 0.294 \\
FB Sentiment & 5 & 0.87 & 0.502 \\
PC3 (Sentiment Factor) & 5 & 1.45 & 0.206 \\
\bottomrule
\end{tabular}
\begin{tablenotes}
\small
\item Tests conducted on full sample with VIX and sentiment lags 1--5.
\end{tablenotes}
\end{table}

\subsection{Coefficient Stability}

Sentiment factor coefficients show general stability across most cross-validation folds, with increased volatility in the most recent periods (folds 12--14). The PC3 coefficient remains consistently positive across all 10 folds at the 1-day horizon (Figure~\ref{fig:stability}), indicating robust predictive contribution rather than a single-period artifact.

\section{Interpretation}

\subsection{Why Sentiment Helps at Short Horizons}

Sentiment likely captures two effects at short horizons:

\begin{enumerate}
    \item \textbf{Attention effects}: Negative news may heighten investor attention, increasing short-term volatility through order flow and hedging activity
    \item \textbf{Information content}: Sentiment may proxy for unobserved fundamentals (earnings expectations, supply chain disruptions) that affect near-term uncertainty
\end{enumerate}

The 1-day horizon is where these effects are most pronounced---before they are fully incorporated into VIX levels.

\subsection{Why HAR-IV Dominates at 5 Days}

At weekly horizons, volatility persistence dominates. The HAR-IV model captures the well-documented mean-reversion and clustering properties of volatility. Sentiment effects at this horizon are either:
\begin{itemize}
    \item Too transient (already priced in within 1--2 days)
    \item Too slow-moving (only visible at monthly horizons)
\end{itemize}

This ``awkward middle'' is a common finding in volatility forecasting literature. The slight underperformance at 5 days, while statistically insignificant, suggests that sentiment may introduce noise at this particular horizon.

\subsection{Why 22-Day Improvement Exists}

The modest improvement at monthly horizons may reflect slow-moving sentiment trends capturing structural shifts in market uncertainty. Monthly reporting cycles and gradual information diffusion could explain why sentiment matters at this horizon but not at weekly frequencies.

\subsection{Why Lack of Granger Causality $\neq$ Lack of Predictive Value}

Granger causality tests require sentiment to contain unique information beyond the full history of VIX and other variables. Our non-significant Granger results mean:

\begin{itemize}
    \item Sentiment does not Granger-cause VIX in a strict econometric sense
    \item However, sentiment \textbf{improves forecasts} when combined with HAR-IV features
\end{itemize}

This distinction is important: forecasting is about predictive accuracy, not causal identification. Our contribution is to the former, not the latter.

\subsection{Economic Interpretation}

A 14.97\% RMSE improvement (0.35 VIX points) translates to economically meaningful value. For VIX options trading, this reduction in forecast error can improve hedging effectiveness and reduce the cost of volatility protection. At typical VIX levels around 15--20, a 0.35-point improvement represents approximately 2\% of the VIX level, which is significant for short-term trading strategies.

\subsection{Practical Implications}

For volatility traders:
\begin{itemize}
    \item Sentiment provides modest but significant edge at daily frequencies
    \item At weekly horizons, simple HAR-IV suffices
    \item Sentiment signals should be treated as supplementary, not primary, inputs
\end{itemize}

\section{Conclusion}

We demonstrate that news sentiment contains economically and statistically significant predictive information for VIX at short horizons. Using a methodologically rigorous approach---rolling PCA, proper out-of-sample evaluation, and horizon-adjusted inference---we find that augmenting HAR-IV with sentiment factors reduces 1-day forecast RMSE by 14.97\%.

Our methodological contributions include:
\begin{enumerate}
    \item \textbf{Rolling PCA} to avoid look-ahead bias in factor extraction
    \item \textbf{Horizon-adjusted inference} for multi-step forecasts using Newey-West variance
    \item \textbf{Proper orthogonalization} of sentiment to returns via residualization
\end{enumerate}

Our contribution is not to claim that sentiment causes volatility, but rather that it provides incremental predictive value when properly extracted and combined with traditional volatility factors. The limited improvement at longer horizons, and the explicit acknowledgment of 5-day underperformance (slight, statistically insignificant), reflects our commitment to honest reporting over result optimization.

\paragraph{Replication} Python code for data collection, sentiment processing, rolling PCA, and forecasting is available at the project repository. All data sources are publicly available as detailed in Section 2.1. Complete replication materials, including frozen output files and validation scripts, ensure full reproducibility of our results.

\bibliographystyle{apalike}
\bibliography{references}

\newpage

\section*{Figures}

\begin{figure}[H]
\centering
\includegraphics[width=0.9\textwidth]{FINAL_RESULTS/figures/figure3_pca_loadings_pub.png}
\caption{PCA Component Loadings (Rolling-Averaged)}
\label{fig:pca_loadings}
\begin{figurenotes}
\small
Loadings averaged across 10 cross-validation folds. Error bars show $\pm 1$ standard deviation. PC3 clearly captures sentiment variation (AV and FB sentiment load at $-0.54$), orthogonal to return (PC1) and volatility (PC2) factors.
\end{figurenotes}
\end{figure}

\begin{figure}[H]
\centering
\includegraphics[width=0.9\textwidth]{FINAL_RESULTS/figures/figure1_forecast_vs_actual_1d_improved.png}
\caption{1-Day Ahead VIX Forecasts: Model Comparison}
\label{fig:forecast_comparison}
\begin{figurenotes}
\small
Out-of-sample forecasts for 1-day horizon. HAR-IV RMSE: 2.338. PCA+Sent RMSE: 1.988. Improvement: $+14.97\%$ ($p < 0.001$). Shaded regions indicate periods where the augmented model outperforms baseline.
\end{figurenotes}
\end{figure}

\begin{figure}[H]
\centering
\includegraphics[width=0.9\textwidth]{FINAL_RESULTS/figures/figure2_rmse_comparison_pub.png}
\caption{RMSE by Forecast Horizon}
\label{fig:rmse_comparison}
\begin{figurenotes}
\small
RMSE comparison across three forecast horizons. Only the 1-day horizon shows statistically significant improvement. The 5-day horizon shows slight underperformance (statistically insignificant), while the 22-day horizon shows marginal improvement ($p = 0.051$).
\end{figurenotes}
\end{figure}

\begin{figure}[H]
\centering
\includegraphics[width=0.9\textwidth]{FINAL_RESULTS/figures/figure4_sentiment_stability_pub.png}
\caption{Sentiment Factor Coefficient Stability}
\label{fig:stability}
\begin{figurenotes}
\small
Sentiment factor coefficients show general stability across most cross-validation folds, with increased volatility in the most recent periods (folds 12--14). The PC3 coefficient remains consistently positive at the 1-day horizon, indicating robust predictive contribution.
\end{figurenotes}
\end{figure}

\end{document}
